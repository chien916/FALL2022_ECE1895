\documentclass{article}
\usepackage{graphicx}
\title{ECE 1895 - ASSIGNMENT 1 REPORT}
\author{Yinhao Qian}
\begin{document}
	\maketitle
	If pictures are obscure, they can be found from /LATEX. 
	Calculation source codes can be found under /MATLAB.
	Schematics can be found under /LTSPICE.
	\section*{Values selected:}
	According to the data sheet and and the relationships between resistors and periods, minimizing $R_A$ yields $t_h=t_l$. I selected the following values because such resistors are common, and making sure $R_A$ is very small compared to $R_B$:
	\[R_A=100\Omega\]
	\[R_B=10K\Omega\]
	\begin{verbatim}
		%Inputs:
		val_RA = 0.1e3;
		val_RB = 10e3;
		val_pRan = [20e-6,500e-6]; %period range
		%Calculations:
		val_CRan = val_pRan/(0.693*(val_RA+2*val_RB))
	\end{verbatim}
	\[C\subset(1.4358\mu F,35.8956\mu F)\]
	Picking a common capacitance values:
	\[C=10\mu F\]
	All other capacitors and resistors have no effect on the period, so I'll leave them unchanged from the data sheet. For the source voltage,however, since no peak voltages are specified as per the requirements, I'll select an arbitrary source voltage:
	\[V_{CC}=10V\]
	All other required calculations are as follows:
	\begin{verbatim}
		val_C = 10e-9;
		val_peri = 0.693*(val_RA+2*val_RB)*val_C %period
		val_freq = 1.44/((val_RA+2*val_RB)*val_C) %frequency
		val_oddc = val_RB/(val_RA+2*val_RB) %output driver duty cycle
		val_owdc = 1 - val_RB/(val_RA+2*val_RB) %output waveform duty cycle
		val_lthr = val_RB/(val_RA+val_RB) %low-to-high ratio
	\end{verbatim}
	Results:
	\[\mbox{Period}=139.293\mu s\]
	\[\mbox{Frequency}=7.164KHz\]
	\[\mbox{Output Driver Duty Cycle}=49.75\%\]
	\[\mbox{Output Waveform Duty Cycle}=50.25\%\]
	\[\mbox{Low-to-high Ratio}=99.01\%\]
	\section*{Schematics}
	\includegraphics[width=\columnwidth]{schmatics.PNG}
	\section*{Simulations}
	Note that the period corresponds with what we desires.
	
	\includegraphics[width=\columnwidth]{waveform.PNG}
	
\end{document}